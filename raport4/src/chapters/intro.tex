\chapter{Introduction}
\label{chapter:intro}

\section{Report Overview}
\label{sec:intro-section1}
In this report we aim to prove the advantages of using Hyperledger Fabric by building a supply chain system that is transparent for all the entities involved in this process and to reduce the time consumption of the acquisition procedures.
Ixia has an exhausting CAPEX process for acquiring the hardware devices needed by the development and QA teams so we believe that building a system based on Blockchain that solves the transparency and bottleneck issues is the perfect fit for replacing the existing procedures.
 
\subsection{Project Description}
\label{sub-sec:intro-subsection1}
Hyperledger Fabric is a opensource framework developed by IBM and supported by the Linux Foundation that offers a permissioned private Blockchain solution. 
The framework is developed for private companies that know exactly the type of entities that are involved in their processes and what are their roles.
Therefore the Hyperledger Fabric is suitable for solving the hardware devices supply chain transparency and bottleneck issues and to restrict the access of the participants involved in accordance with their position in the company. The hardware supply chain is CAPEX process that requires special approvals and priority analysis which is time consuming and often hard to track.


\subsection{Project Objectives and Motivation}
\label{sub-sec:intro-subsection2}
Hardware supply is an important aspect of the Ixia's employee daily work. Many software applications developed by Ixia teams run on dedicated hardware and in many situations only one employee can use that hardware at a time.
The existing CAPEX procedure is very complicated and the time between ordering a new device and receiving it can be really long. It may happen that the team doesn't need it anymore at the time the device was received. The dedicated hardware is expensive (some chassis may cost up to \$50000) so not receiving it on time may cost the department a lot of money and it can also delay the releases.

On each quarter, the manager has the responsibility of determining what are the hardware needs of the team and send the list to her/his superior for approval.
Also, the manager has to check the price of the devices requested and if the costs are over the CAPEX budget, the hardware list should be prioritized. 
After completing the list and set the priorities if needed, the manager has to fill an \emph{Internal Sales Order} on an internal platform. 
The hardware manufacturing managers have to check if there're are new ISOs for their teams to build or if they have the devices on stock. If the hardware machines are on stock they will be delivered to the department/manager that requested them. If there's no hardware with the desired specification on stock, the manager has to schedule it for manufacturing.
Also, the manufacturing managers need to make sure that the financial transaction was done successfully before delivering the hardware so they have to discuss with the financial department and validate that everything is ok.
A huge disadvantage of this procedure is the lack of transparency regarding the status of the order and the time estimation until receiving the order.
In addition to this the costs may differ due to the international tax changes. 

The \textbf{Hardware Supply Chain on Blockchain} project is a supply business network that solves both the transparency and the access' restrictions.
This system is using Hyperledger Composer framework for modeling the business network: participants, transaction and assets and connects to the Hyperledger Fabric via its API to add transactions in the Blockchain database and to manage participants's access accordingly with their roles in the network.


%\ref{sub-sec:intro-subsection2}
%\textbf{\project} 
%\begin{itemize}
%\item 
%\end{itemize}

%\labelindexref{Figure}{img:modularIUmbrella}.
%\fig[scale=0.7]{src/img/modularIUmbrella.jpg}{img:modularIUmbrella}{Hyperledger Modular Umbrella Approach(source \cite{edXTut})} 












