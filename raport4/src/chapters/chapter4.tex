\chapter{Hardware Supply Chain}
\label{chapter:chapter4}

\section{ISG CAPEX Procedure}
\label{sub-sec:chapter4-section1}
The \emph{\project} use cases are ordering a new hardware device or returning the merchandise authorization(RMA).
ISG(Ixia Solution Group) teams are focused on developing software solutions starting with testing the network performance, getting network visibility by monitoring the traffic or finding security threats before compromising any data.
Therefore teams need specialized hardware devices such as routers, gateways, switches to more powerful machines as packet brokers, ESXI servers, clusters, etc.
The ISG division offers deep expertise in network testing, security and visibility which requires investing in powerful hardware equipment.
In each quarter of the financial year the ISG division has a CAPEX budget to invest in hardware so each team can request new hardware devices.
Each manager has the responsibility of identifying the hardware requirements by consulting his/her team and assuring that the total cost doesn't exceed the budget in which case the devices list should be prioritized.

When the CAPEX requirements list is completed, the manager can proceed to order them.
For ordering the equipment he/she will use the \emph{\project} system.

The RMA procedure can be performed by the team manager as soon as he/she is informed about the hardware malfunction. Most hardware devices have an warranty period where they can be replaced or repaired free of charges by the manufacturing divisions. 

\section{Use Case Description}
\label{sub-sec:chapter4-section1}




