\chapter{Introduction}
\label{chapter:intro}

\section{Project Description}
\label{sec:intro-section1}
The research report presents the actual state of Blockchain technology and existing projects that aims to offer Blockchain based solution for businesses around the world. 
The report identifies the areas that will benefit most from the adoption of Blockchain and how they will look like after the switch. Another important aspect is the perception of business leaders when it comes to Blockchain technologies and what are their concerns related with these technologies.
In order to be able to do such an analysis it’s necessary to study the existing Blockchain businesses and their history from an idea to a successful startup. After determining the keys elements that made these businesses so popular we focus our attention to Decision Support System architecture and the existing solutions. Last but not least, we are going to describe what are the main advantages of the system proposed and what makes it better than the other available solutions.

\subsection{Brief Project Description}
\label{sub-sec:intro-subsection1}
\textbf{\project} is a decision support system that enables leaders to envision how Blockchain technologies will revolutionize the industry they activate in.
Another main aspect is that the system is going to work as an extension of Hyperledger Composer framework, therefore, the modeling format will be compatible with Hyperledger Composer Model Files.
In addition, the system offers support to determine whether or not the targeted business is suitable or not for Blockchain switch.


\subsection{Project Scope}
\label{sub-sec:intro-subsection2}
The project aims to achieve the following goals:
\begin{itemize}
	\item to improve the existing modeling solutions by reducing the time and costs of design a Blockchain solution from scratch;
	\item to keep the compatility with other projects that have the same scope;
	\item to determine the fields that could be innovated by the adoption of Blockchain;
	\item to understand the needs and limitations of the existing solutions in order to offer the best version of a DSS.
\end{itemize}  

\subsection{Motivation and Objectives}
\label{sub-sec:intro-subsection3}
Blockchain represents one of the five digital transformations in the I\&T world along with banking, smart contracts, connected cars, healthcare and IoT.
Developing a Blockchain based application requires knowledge from different fields from networking, protocols, cryptography, encoding, digital signatures, algorithms, etc. Therefore, developing a Blockchain system can be very complicated and complex even for experts and professionists in the field.

In addition, innovation is about building upon existing technologies and take advantage of the already knowledge of the communities so developing a Blockchain based solution from scratch is not necessary anymore. 

Another reason why business people and decision makers are sceptical is the fact that traditional Blockchain projects allows anyone to connect to the network and their identity cannot be verified. 
Starting from these perspectives, we have decided to implement a blockchain solution that is using a powerful framework developed by the Linux Foundation, Hyperledger Project.
Hyperledger Project is offering a solution to all the issues metionated above. The idea behind Hyperledger is to develop and nurture an ecosystem for the future of business blockchain technologies.
Hyperledger represents an alternative to the cryptocurrency-based blockchain model, and it offers blockchain frameworks and modules to support global enterprise solutions. The main focus of Hyperledger is to provide a transparent and collaborative approach for Blockchain development community.

The purpose of this research paper is to identify ways of helping leaders around the world to innovate their business using blockchain technologies. 
It is easy to see the huge potential of Blockchain. Business leaders and organizations around the world believe that it will have the same impact on transactions as internet had on communications. Blockchain is still an young technology and its potential is yet to be discovered.
Taking this information into consideration, we believe that a prototype system that enables leaders to visualize how their business will look after the adoption of Blockchain technologies will accelerate this switch.

Customers come first so we need to think about creating value for the customer through technologies. Leaders need to be open and aware of the transformation that happens around them. Therefore, we want to build a system that offers support for decision makers and business leaders to envision the future of their organization after the Blockchain switch. 
After doing some research on this direction, we discovered the existence of a Hyperledger framework, the Hyperledger Composer that creates Blockchain solutions for organizations with no expertise or knowledge of this technology.
