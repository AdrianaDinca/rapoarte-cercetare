\chapter{Background}
\label{chapter:chapter1}

\section{Blockchain Technology}
\label{sec:chapter1-section1}

Blockchain is a concept that offers support for a peer-to-peer electronic cash system that is able to solve the classic issues of digital money peer-based systems. The biggest challenge was the double spending issue that was solved by introducing a chain of blocks with transactions where each transaction/block has a unique hash computed using a proof-of-work algorithm. In order to add a transaction into a block it has to be validated by the majority of nodes to ensure the unique hash wasn't never added in the chain. After this validation, the transaction is added in the block and all the chains of the network update their database with this new transaction. 

The proof-of-work algorithm is a hash function that requires a lot of CPU power in order to obtain a new unique hash. Thus, if an attacker tries to double spend its money it has to compute 2 hashes sequentially: the first hash is computed using the longest chain's proof-of-work and the second hash should be based on the new longest chain which already has the information about the money spend in the first transaction and as a result the network nodes are going to invalidate the second transaction. The problem still can appear if the attacker holds the majority of nodes by adding the second transaction in the chain. The possibility of having a monopoly on the system is really small if we think that it requires a lot of money to keep the nodes up and running. Additionally anyone can join the network from everywhere around the world with no real connection to the other members so this enables a wide diversity of individuals with different backgrounds and conceptions.

A node can leave the network and return to it at any time by updating its Blockchain history with the newest transactions since its inactivity so the network broadcasts the protocol messages only to the nodes that are active. When a node rejoins the network it will add the already validated transactions in its chain. The blocks' chain of transactions holds immutable information, it is the proof of all the events happened in the network so it is called the Blockchain's ledger.

The Blockchain idea was defined by Satoshi Nakamoto in his white paper called \emph{Bitcoin: A peer-to-peer electronic cash  system} \cite{bitcoinwhitepaper} as a electronic payments system that uses the Bitcoin digital currency to facilitate financial transfers between peers without the involvement of a central authority.
This idea solves the Byzantine Generals Problem using proof-of-work for achieving consensus. Jaroen RijnBout \cite{pow-byzantine} describes how the Byzantine Generals Problem is solved using the consensus protocol proposed by Satoshi Nakamoto. He presents the story of many generals that want to crack the king's Wireless password and in order to achieve success the majority of generals must contribute to generate the CPU power at the same time to brute force the password. Thus, we need to specify a time of attack and check if the majority of generals agree with the plan. For setting the time of attack, all the generals propose a time and compute a transaction hash. The generals can agree or not on the proposal so they can decide to add it to the ledger or not. We know that an hour of CPU power is enough to brute force the king's password and the time to compute the hash for a transaction is ten minutes if (and only if) all the generals work in the same time. Therefore, we need six generals to attack at the same time for the attempt to be successful. In order to determine if enough generals agree with the attack we need a chain of transactions that starts with the transaction that has the proposal and other six transactions added sequentially. Therefore, each general that agrees with the attack can see if the number of generals that want to attack is big enough so their attempt could succeed.

More than that, the Blockchain protocol removes the central authority by building a general solution to the Byzantin Generals Problem that allows members to reach consensus about the transactions' order.
Satoshi's idea is considered the biggest revolution since the appearance of Internet and its potential goes beyond the financial world. It has the power to change the business world by innovating in the fields of supply-chain communication and provenance, cloud storage, manufacturing, electronic voting, employee payments, etc.
In the following sections we are going to describe the business areas that can benefit from Blockchain and the way these changes can be accomplished along with the existing projects involved in this revolution.
The technical details about the protocol are really well defined by Satoshi in his white paper \cite{bitcoinwhitepaper} so our aim is to identify the main directions of applicability of the Blockchain technology with both strong and weak points. 

\section{Blockchain for Business}
\label{sec:chapter1-sec2}
The Blockchain technology is an young area of research and the range of industries that can take advantages of this protocol is wide but we manage to determine the main directions of applicability. Before describing each direction it is important to briefly present the new features and concepts that were introduced after the Bitcoin public release.

The popularity of this technology has expanded rapidly and many specialists named it the \emph{Trust Protocol}. It is the only system that solves the Third Party Issue via a fully decentralized network where each participant has equal rights of voting or mining. The process of mining refers to the process of value generating via digital currency. Each block contains a new coin transaction as the first transaction added in the block. In this way, the amount of money of the network is constantly increasing. The value of one coin is directly proportional with the amount of CPU power and time consumed to generate that coin.
In a network the miners sets the fees for network's infrastructure and adjust the network traffic related parameters based on traffic analysis. In addition, there are no fees for transactions and the infrastructure fees are considerably lower than any bank or financial institution's fees.

The data stored in the ledger is encrypted so there's no way to changed it so the system can be used as a decentralized storage application.

On the other hand, the system offers transparency so all members of the network can see any transaction stored in the ledger with the mention that the author of the transaction is anonymous. All members need a public-private key to send and receive money and only the public key can be seen by the other members without any association with a real person. For public companies this level of transparency can affect their business and many of them prefer to use the classic payment system. Anyway, a transparent system offers accountability and there are areas of industries and governmental institutions that can solve the trust issue by adopting this technology. 

Also, the Blockchain system is immutable so once data was added in the ledger it cannot be changed. This feature is accomplished via a cryptographic hashing algorithm. For any data that is added in the database, a hash is computed with a fixed length. The hash has double functionality: reduces the size of any data to a fixed value and it is used by the database chain structure (a linked list) so each block has a hash pointer that references the previous block from the list and so on. In this way, only a minor change in the data will generate a completely different hash so all the previous blocks and links from the list should be regenerated. Any attempt to corrupt the ledger data has proven to be inefficient, the costs of regenerating the ledger are huge and it is quite absurd to waste CPU power in this direction while you could use it for mining.

In the first decentralized Blockchain based networks the nodes have equal rights in the process of decision. The benefit of this approach is the removal of any hierarchical structure that could undermine the wish of the majority of participants. These systems were designed to give their members equal power and to offer an alternative to a wide range of industries that use centralized systems such as governmental institutions, banks or private companies. Although these systems have a flat topology members can have different roles and their level of involvement can be different. The main limitation of this approach is related with scalability. In order to send data to all the network nodes, every node that receives a message from one of its neighbors, has to send it to all other neighbors. This mechanism is similar with gossip and has the disadvantage that it affects the network performance. As a solution to these problems, the new Blockchain based systems decided to designate some nodes to be node \emph{leaders} that have more decision power in return to spreading the information faster to all the members of the network. This solution has the disadvantage that it is not fully decentralized but it can be a good option in situations where speed is more important than decentralization.

In 2019 the main applicability field of Blockchain remains finance. Big financial organizations started to work on projects that use this protocol. For example, US Federal Reserve is working on creating a new digital payment system that uses Blockchain and the Nasdaq stock exchange is using a Blockchain ledger - Nasdaq Linq to record shares transfers in the private market.

However, the areas where Blockchain can innovate exceed the financial world. The Blockgeeks community \cite{blockgeeks} affirms that Blockchain and the Web 3.0 will offer users the support to \emph{create value and authenticate digital information}.
Blockgeeks identifies the industries that will benefit from the adoption of Blockchain as follows:
\begin{itemize}
	\item Smart Contracts - computer programs that define the details of a contract between network members and get executed when all the required conditions are met;
	\item Sharing Economy - a peer-to-peer network that enables good and services sharing without involving a third party;
	\item Crowd Funding - raising crowd-sourced venture capital using decentralized networks offers high transparency;
	\item Governance - transparent and private systems for voting and polling, smart contracts for managing administrative prerequisites and results;
	\item Supply-chain Auditing - Blockchain ledger can be used as a tracking and audit tool for verification about the origin and intermediaries steps from a product life-cycle;
	\item File Storage - a decentralized system solve the data corruption or lost; the face of Internet can be completed changed by replacing client/server web applications with decentralized systems;
	\item Prediction Markets - predict events using the crowd knowledge via a decentralized network;
	\item Intellectual Property - use smart contracts to protect copyrights and define the intellectual property market in a decentralized manner;
	\item Internet of Things - use smart contracts to manage the IoT systems by collecting their data and to make decisions and predictions by analyzing IoT components state;
	\item Microgrids - use smart contracts to sell/buy the excess of solar energy in areas close to the source point; these applications can automate the process of computing the amount of energy required and determine the cost of the energy dynamically using a IoT monitoring systems; 
	\item Identity Management - building a global identity management system is a complex task but distributed systems can solve it; the Blockchain ledger can store also individual information such as reputation which is really important for e-commerce relationships. 
	\item AML and KYC - refer to the anti-money laundering and know-your-customer problems that can be reduced using a cross-company ledger; it involves smaller costs for customer data analysis and creating reports of suspicious transactions;
	\item Data Management - building a decentralized personal data marketplace; users will collect the profit of selling their personal data to companies for data analytics and statistics.
	\item Land Title Registration - reduce fraud and managerial cost of registering land titles via a Blockchain ledger that guarantees immutability and transparency;
	\item Share Trading - improve the share arrangement process by reducing the confirmation time using a peer-to-peer network and removing any intermediaries from the arrangement;
\end{itemize}

Taking all these information into consideration, we can predict a completed different business world, where the majority makes the rules. With Blockchain we don't need banks to take care of our transactions, no more unreasonable fees on any operation. For example, a airline company can offer a flight ticketing service on Blockchain that with remove the costs associated with the online payment fees both for them and their customers. Another good example is the car sharing companies that are no longer needed if a driver can contact his clients directly in a Blockchain based network and the company payment side will go into the customer pocket. The list of use cases is huge from auction companies to all types of intermediary business. Also this technology can make some industries areas profitable again. The music singers and film makers can win more money by removal of music and streaming distributors.

Another great futurists Don Tapscott and Alex Tapscott \cite{tapscott} affirm in their book that \emph{the technology likely to have the greatest impact on the future of the world economy has arrived, and it’s not self-driving cars, solar energy, or artificial intelligence. It’s called the blockchain}. They tried to envision the future of economy in the short, medium and long term and the results are are quite impressive. They have identified seven generic areas that would be transformed via this technology as follows:

\textbf{Financial services} is the first area that can be transformed by the Blockchain protocol. The principal aspects impacted by this protocol are:
\begin{itemize}
	\item \emph{secure cryptographic identities} that can be easily verified can interest the rating agencies, marketing and customer's data analysis organizations, regulators, etc;
	\item \emph{value's transfer} without intermediaries will reduce the costs of buying/selling goods and services and this aspect can interest commercial banks, card networks, telecommunication companies and regulators;
	\item \emph{value' storage} refers to new ways of storing any type of value from financial instruments, money market funds and government securities; brokerage agencies, retail and investment banks, regulators and telecommunication companies could be some of the interested parts of this transition;
	\item \emph{value's loan} is any type of credit from credit cards, mortgages to governmental bonds; it can be issued, traded and paid and settled in the Blockchain ledger which will improve the efficiency and reduce the systematic risks;
	\item \emph{value's trading} enables investments banks, brokerage agencies, investors, central banks, future contracts organizations and regulators to reduce the time of transactions' settlements from days or weeks to seconds or minutes; this time reduction offers enrichment's opportunities to individual without access or with limited access to financial services;
	\item \emph{funding and investing in an active} via a peer-to-peer network offer investments banks, legal and audit companies and stock exchanges the possibility to automate their dividends by smart contracts execution;
	\item \emph{value's insurance and risk management} are more transparent by using decentralized markets so entities such insurance companies, risk management companies, brokerage agencies and regulators can better estimate the risks;
	\item \emph{value's accounting} can be solved via smart contracts so all type of audits and accounting reporting can be generated in real time and it guarantees a high level of accuracy and transparency so regulators and audit companies can improve their activity.
\end{itemize}

Another area that can be transformed by the Blockchain technology is \textbf{the organization} by completely redesigning its architecture. The two futurists Alex and Don Tapscott describe an organizational structure that is decentralized, where managers have only a consultative role and the decisions are taken by the organization's members which are also owners. Members can work on one or more projects at a time and they get rewarded based on their contribution in the company. An interesting example is the \emph{ConsenSys} company that is one of the pioneers of Ethereum software development. This company focuses also in building an new type of organization based on cooperation not on hierarchy where the classical jobs description is replaced with dynamic roles, the authority is distributed along the members of the organization, the rules applied are transparent and the big organizational structural changes are replaced by fast and continuous reiterations.

The \textbf{new business models} that have the potential to generate wealth are using the concept of distributed applications (dApps); starting from this idea we can identify four business models: 
\begin{itemize}
	\item \emph{smart contracts} are a business model that has one functionality so its level of complexity is low and they are automated so there is no need for human intervention; the human support is replaced by multiple digital signatures agreements;
	\item \emph{open network enterprise} is a business model with a low level of automation and a high level of technical complexity by allowing smart contracts to interact with each other and create new intelligent business decisions; in this model business is similar with networks so the organization's borders will expand;
	\item \emph{autonomous agents} are a business model with a low level of automation so they require human intervention and a low level of complexity by having one functionality; in our model it represents a soft that can extract information from the environment on behalf of its owner and has the ability to make decisions independently;
	\item \emph{autonomous distributed enterprise} is a business model with high levels of complexity and automation; they are the result of combining autonomous agents with open network enterprises; they are dynamic ecosystems with no hierarchy that aims to produce customer's value and owner's wealth.
\end{itemize}
The open network enterprise is a feasible business model that can replace the centralized business models that has the potential to innovate the business world and to offer better services and products at a lower cost. From this category we can identify the peer producers (members can build their reputation using the Blockchain technology both in public and private space and be rewarded accordingly), copyright owners (members can protect their intellectual property by storing the data encrypted in the Blockchain and sell it directly), cooperatives Blockchain systems (the goods and services creators can generate more value by removing the intermediaries' side), counted/sharing economy (members can rent or sell any unnecessary goods and services), Blockchain creators (members that brings data in the ledger in order to keep track of every product from its production to the final consumer) and enterprise collaborators (members can collaborate with their knowledge to building value; for example, a social Blockchain enterprise may benefit both a user that is interested to earn money from selling some of its information such as its food preferences and for a food company that wants to launch a new  type of food).

\textbf{Registry of Things} represents a way to share information in a distributed and secure manner, automate transactions and actions in Internet using the Blockchain technology. Its ledger is a immutable registry of all data exchanges that happen in the network so members benefit from trustworthy information. The Internet of Things can use Blockchain to store its data collections in the ledger, communicate with other IoT systems and define smart contracts to automate any agreement or payment. For example, we can define a smart contract that includes some limitations of intensive agriculture and if a farmer makes an abuse the IoT system that monitors its agricultural activity can inform authorities about the abuse.

An approach of \textbf{solving the prosperity paradox} is also connected with the Blockchain protocol. In the last decades we can observe that the gap between the rich and poor people has increased worldwide and less 1\% of the global population owns more than 50\% of world's wealth. This economic inequality is due to the financial exclusion: 15\% of the OECD\footnote{Organization for Economic Co-operation and Development - international organization  with 36 member countries which focuses on building better policies to improve the lives of people around the world \cite{oecd}} countries' population never worked with a financial institution and 73\% of Mexico population cannot access any banking service. For these people, the Blockchain is the only way to create a financial identity and start using banking services such as accessing loans for investments. The amount of money the Blockchain members can offer to a new financial identity can constantly increase if the member pays its debt on time and his reputation is improving.

\textbf{Rebuilding the democracy and the governance} is also an astringent problem of the democratic states where there is a big difference between the theory and how politicians apply the democratic principals. Many Blockchain adepts affirmed they are libertarians and anarcho-capitalists so the idea of using a decentralized system to governance over the public space and to fulfil the wish of the people wasn't really well received by the governance. By the adoption of Blockchain we can image a nation that have high performance governmental services and operations where the power is granted based on the politician ability to serve other people. Taking these ideas into consideration we can present a new historical age for democracy with electronic voting on Blockchain, new models to do politics and justice where all citizens get involved in solving their community biggest problems.

The \textbf{cultural life} is also a interesting area that can be innovated by the use of Blockchain. For example, this technology can restore musicians the possibility to distribute their music directly to their fans without paying high fees to other companies. Another use case related with this area is the intellectual property copyright so the owners can save their work encrypted on the Blockchain and sell it directly to their fans. The Blockchain can be a solution for achieving freedom of the press and freedom of expression in a world where information can be manipulated by its legal protectors due to personal fears and 
external pressures.

In conclusion, the Blockchain technology has the potential to revolutionize a wide range of industries and areas of day-to-day life. At this moment, it is an area of research and we aim to contribute to it by describing the state of the art of Blockchain in the business world and building a proof-of-concept using some assistive frameworks and tools for Blockchain based systems development.








 

