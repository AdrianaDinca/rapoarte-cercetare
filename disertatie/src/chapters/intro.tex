\chapter{Introduction}
\label{chapter:intro}

This thesis presents the Blockchain technology from a business perspective as an attempt to identify the main areas of applicability for the business world.
In order to validate the information gathered in the state of the art and research stages we build a Blockchain based system in the area of hardware supply-chain for the CapEx~\footnote{Capital Expense - the budget allocated by a company to buy new assets or replace the malfunctioned ones} process of the NVS~\footnote{Network Visibility Solutions} department of ISG~\footnote{Ixia Solutions Group}, Keysight Technologies. The content of this thesis outlines the benefits and limitations of this protocol and how the business world can revolutionize its processes by adopting it.

In this chapter we describe briefly the current state of Blockchain for business and how it can solve the biggest problems of Internet. Furthermore, we are going to outline what are our motivations for building a hardware supply-chain using Hyperledger frameworks and what objectives we want to accomplish as well as our contributions to the scientific community.

\section{State of the art}
\label{sec:intro-section1}
Alex Tapscott and Don Tapscott \cite{tapscott} affirm that in business we need to follow the four basic principals of integrity: honesty, responsibility, transparency and consideration. Before the Blockchain innovation, the digital world relied exclusively on third party entities for certifying a business partner. The Blockchain is considered to be the only solution of the third-party issue so it is called the \emph{Trust Protocol}. This technology has an enormous potential to revolutionize the Internet by offering it the integrity level required for business operations.
Blockchain offers a new prosperity plan for peer-to-peer economy, speeding the process of financial inclusion, global protection for economic rights, eliminating bureaucracy and corruption from foreign aid, rewarding value creators, corporation redesign as the motor of capitalism, animate the IoT via collaboration, building a new Blockchain entrepreneurship, etc.

Many companies and governmental organizations consider this technology a threat because of its power to close down industries such as banks and to create a new democracy by equal distribution of decision power. In its early stage the Blockchain was associated with the black market so it offered its opponents a reason to undermine it.

However, the Blockchain's enthusiasts contend its potentially dangerous aspects and they started to build Blockchain based applications for e-commerce, electronic voting systems, digital identity, financial transaction, business collaboration, etc. For supporting the growth of Blockchain networks, the online community developed solutions to help entrepreneurs adopt the protocol in their business. 
Hyperledger is a multi-projects organization that focuses on developing frameworks and tools to assist enterprises to integrate this protocol in their business.
Fabric is the most popular project developed by Hyperledger. It offers a modular Blockchain implementation for cross-industry networks that provides an authorization layer to enable access only to trustworthy members. 


\section{Motivations and Objectives}
\label{sec:intro-section2}
We were motivated to choose this direction of research because it is an young technology with multiple areas of applicability for business and it offers solutions to building a new Internet where trust is embedded. As a result of this research, we reached the conclusion that systematizing the Blockchain opportunities for the business world is valuable and based on this we build some short and long term objectives:

\paragraph{$\bullet$ Exploration stage:} analyze the existing Blockchain for business projects and identify the main industries  where this protocol can revolutionize the business operations; describe the reasons why Blockchain can change the business world; outline the main characteristics of this technology that enables entrepreneurs to completely redesign their organization; systematize the business areas that can be innovated;
	
\paragraph{$\bullet$ Evaluation stage:} explore Blockchain based projects to identify the best approaches to build such a system and test the most popular technologies and frameworks that offer transparency, accountability and decentralization; discover the Hyperledger community and test some of its solutions for developing Blockchain based applications from scratch without advanced technical skills; the Hyperledger organization aims to support cross-industry Blockchain networks and it offers assistance for new Blockchain ideas for enterprises;
	
\paragraph{$\bullet$ Validation stage:} validate that the adoption of the Blockchain technology in business doesn't required highly-skilled professionals by using open-source assistive tools so we build a Blockchain based application using two of the Hyperledger projects: Fabric and Composer; propose a network for CapEx business operations for ISG that solves the accountability, transparency and authorization issues of the current procedure; this system is based on a business model that define the participants, their role in the CapEx process, the type of operations associated with it and a business logic for every operation involved in this procedure.

\section{Contributions}
\label{sec:intro-sec3}
This thesis of Blockchain for business was a tough challenge from the point of view of variety - there are a wide range of Blockchain based projects that aim to innovate the business world so the amount of information is quite vast and not everything is valid. Firstly, in chapter \ref{chapter:chapter1} we described the business related characteristics of Blockchain and we determine the industries where Blockchain can change the business completely. We systematized all the information gathered and presented two possible classifications: the first one is based on the industry activity and the second goes beyond the current state of business by introducing new business models and empowers everyone to prosper by creating new value or maximizing the profit by eliminating the third party.  

In chapters \ref{chapter:chapter2} and \ref{chapter:chapter3} we presented the Hyperledger implementation of Blockchain and offered a brief description of the most popular projects so we claim our contribution to the Hyperledger's promotion by offering an easy to follow guide to build Blockchain applications using Hyperledger Fabric and Composer frameworks.

Another contribution is represented by the hardware supply-chain network developed as a replacement for the CapEx process for the Network Visibility Solutions department of ISG. The network is using Hyperledger Fabric's Blockchain implementation, a simplified version of the consensus protocol and a membership service module to offer a system that is transparent - everyone can see the transactions issued in the network, participants are certified so the system offers accountability and they have limited access based on their role in the CapEx process. Therefore, we build a proof-of-concept application that solves the business operations issues of CapEx in ISG using the Blockchain protocol advantages without the need to develop a solution from scratch by integrating open-source frameworks.
\section{Summary}
\label{sec:intro-sec4}
The thesis contains six chapters as follows:

\paragraph{$\bullet$ Chapter 1 - Introduction} provides a brief description about the state of the art of this scientific paper by presenting how the Blockchain protocol solves the business issues of the digital world. Also, in this chapter we outline the motivations and objectives of the thesis alongside our contributions to the Blockchain for business' research field.

\paragraph{$\bullet$ Chapter 2 - Background} describes the most important characteristics of the Blockchain protocol and how Blockchain can be used to fulfill business requirements. More than that we outline the two types of Blockchain for business areas types of classification. The first type is oriented to fit the current business environment without radical changes and the second type of classification is more generic and provides insights into the future of business after embedding Blockchain - the \emph{Trust Protocol} into digital business.

\paragraph{$\bullet$ Chapter 3 - Study of Hyperledger Projects} aims to present the Hyperledger organization, its goals and the incubator of Blockchain based projects. Our focus is to describe the Hyperledger Fabric and Composer technical specifications in order to prepare the ground for a better understanding of  the implementation aspects of {\project} network.  

\paragraph{$\bullet$ Chapter 4 - Implementation} offers insights into the design and implementation details of the {\project} system developed in order to outline the usefulness of this protocol in the business world. The application idea originated as a result of a release delay due to insufficient specialized hardware in one of the Network Visibility Solutions projects so we decided to solve the problems of CapEx orders' lack of transparency and no reliable time estimations for orders delivery. All these issues are solved by the {\project} system that uses Hyperledger frameworks and tools.

\paragraph{$\bullet$ Chapter 5 - Hardware Supply-Chain Network} describes the existing CapEx processes and identifies the business operations issues in the current flow of events. In order to prove how our solution fixes these problems we present the main uses case of the system in detail. In order to demonstrate the correctness and completeness of the business network model developed we use the Composer UI for testing called \emph{Playground} \cite{composer-playground}. 

\paragraph{$\bullet$ Chapter 6 - Conclusion} builds the thesis conclusions by summarizing the thesis' contributions for Blockchain for business and suggest other research directions that weren't covered by this paper.


%\ref{sub-sec:intro-subsection2}
%\textbf{\project} 
%\begin{itemize}
%\item 
%\end{itemize}

%\labelindexref{Figure}{img:}.
%\fig[scale=0.7]{src/img/.jpg}{img:}{} 












