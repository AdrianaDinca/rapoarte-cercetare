\chapter{Conclusion}
\label{chapter:conclusion}

This thesis summarizes our contribution to the Blockchain for business research field and proposes future directions of research.
We claim to accomplish our objectives by presenting two possible classifications of Blockchain for business areas and we had identified the most popular Blockchain based projects that are specialized in meeting business requirements.
In chapter \ref{chapter:chapter1} we propose two possible ways of classifying the Blockchain based systems: the first approach is reflecting the current state of the Blockchain for business field while the second approach provides insights in the future of this technology.

In order to validate the assumption that Blockchain technology can be easily adopted by any business we created a proof-of-concept to solve the issues identified in tha CapEx procedures conducted by the Network Visibility Solutions department of Ixia Solutions Group using two business oriented Blockchain based projects: Hyperledger Composer and Fabric. The two frameworks developed by the Hyperledger organization are programmer-friendly so anyone with basic programming skills can manage to develop a Blockchain based application in a few months. In the end, we improved our technical skills with new technologies and we broaden our mind by understanding how economical and social problems can be reduced once Blockchain becomes widely used.

There are still interesting Blockchain for business unexplored directions such as creating a business analysis for every identified area about the state of the current applications and what are the challenges they are facing alongside our solutions to help them grow their business.
Also our proof-of-concept system can be improved by adding a basic web client to complete the Hardware Supply-Chain user experience and extending the network to external organizations in order to obtain a full traceability of the CapEx orders.