\chapter{Study of Hyperledger Projects}
\label{chapter:chapter2}

\section{Hyperledger Project}
\label{sub-sec:chapter1-section1}
Hyperledger is an open source project focused on Blockchain technology that aims to bring IoT, supply chain, finance, banking and manufacturing together. Is is hosted by the Linux Foundation and includes leaders from all the mentioned areas.

The Blockchain is a peer-to-peer distributed network, each participant of the network has its own copy of the ledger, and any transaction is validated by the majority of participants. The validation is done by solving Byzantine Generals Problem via consensus.
The Hyperledger project offers solutions based on Blockchain with additional features such as the \emph{smart contracts} and it has a large number of assistive tools to facilitate the embrace of Blockchain.
All projects developed under the umbrella of Hyperledger and Linux Foundation are applications that use a ledger of transactions to establish transparency, accountability and trust. These projects are for a wide variety of business areas by providing the infrastructure, tools and frameworks to build applications based on Blockchain easily and in a short period of time. Additionally these applications follow the legal constrains by keeping the network closed to authorized participants that can be made accountable for their actions.

The project was launched in 2016 under the guidance of many important corporations such as IBM, Intel, etc. The first codebases that were released to the public were Hyperledger Fabric - a product that was the work result of three organizations (OpenBlockchain from IBM, libconsensus from Blockstream and the smart contracts from Digital Asset) and Hyperledger Sawtooth from Intel. 
These two frameworks were offering support for growing the development of Blockchain business solutions. Later on, there were other projects released that continued to offer assistive tools to support this growth. All projects that were released under the Hyperledger project were supervised by the Hyperledger Technical Steering Committee, a group of eleven specialists that were elected from the technical contributor community.
In May 2016 the Linux Foundation and the corporate members involved in Hyperledger project elected an Executive Director to make sure that this idea has all the resources needed to be successful. In this position was placed the co-founder of Apache Software Foundation - Brian Behlendorf. 
The project became so successful so at the end of the next year the Hyperledger counted seven more projects and the number of members increased to more that 200.

The most successful projects were Hyperledger Sawtooth - multi-language support for distributed ledger, Hyperledger Fabric - the distributed ledger written in GO language, Hyperledger Composer - framework for accelerating the process of developing Blockchain applications, Hyperledger Iroha - the distributed ledger written in C++ and Hyperledger Indy - distributed ledger for decentralized identities.
All these frameworks and tools are used in real-life applications except for Hyperledger Composer which is still in incubation. The Hyperledger Composer is an assistive tool for developers to facilitate the process of building a business network using the Hyperledger Fabric. 
It was proven that Hyperledger Composer is hard to maintain so the Hyperledger board decided to stop the development of new features and to keep the existing functionalities compatible with the new versions of Hyperledger Fabric.
The main focus of the community is to add more features to Hyperledger Fabric framework which has a modular architecture so adding new features can be done quickly and without affecting the pre-existent functionalities.

In the following sections we are going to describe the Hyperledger Fabric and Composer frameworks. The Blockchain technology promises a revolution as big as the Web and the Hyperledger group understood its potential so they formed a Blockchain incubator to offer support to any bright idea related with Blockchain technology, smart contracts and the business world.

\section{Hyperledger Fabric}
\label{sub-sec:chapter1-section2}

%\iffalse
\subsection{General aspects}
\label{sub-sec:chapter1-subsection1}
%\fi
Hyperledger Fabric is an implementation of a distributed ledger developed mostly in Golang. The framework uses also other languages and technologies such as Javascript, Go or Java for smart contracts (chaincode), SDK in Node.js, Python, Java, Go and Rest.
It is a solution for a secure, high-performance and permissioned Blockchain based network that has a modular architecture that allows plugins for any new features with no impact on the core functionalities.

Before getting into more details about the features provided by this framework it is important to mention that it is developed for enterprise use cases so the solution must solve the identity issue for transaction of type know-your-customer or the money laundering issue by offering transparency to authorities as participants with advanced permissions in the network. Thus this framework is permissioned which means that the participants are not anonymous, their identity is known by the other participants so they can be made accountable for their actions. The framework network is private and the transaction content is confidential so it can contain business sensitive information. In \cite{fabric-article}, Elli Androulaki and others are presenting the Hyperledger Fabric as a distributed operating system with all characteristics from the following paragraphs.

More than that the framework offers the  possibility of selecting a consensus protocol based on the business needs. For example, if the business network is used by a single enterprise or it is governed by a authorized identity there's no need for using the Byzantine Fault Tolerant (BFT) protocol so it may sufficient to go with a simplified version such as Crash Fault Tolerant (CFT) consensus protocol. As a result, the Hyperledger Fabric eliminates the low latency of transaction validation and  improves the network performance. The framework doesn't required CPU power for mining or for smart contract execution so the cost is similar with any distributed system.

The Hyperledger Fabric has a wide community of developers and activists that help with the development of new features and maintain the Fabric codebase. The number of contributors has grown to more than 200 members and the organizations involved to 35 so the project benefits of a diverse set of skills. With such a support the future of this project is really promising. It has modular and pluggable architecture so it can bring innovation to many industries.

%\iffalse
\subsection{Technical aspects}
\label{sub-sec:chapter1-subsection2}

Modularity is one of the main characteristics of Hyperledger Fabric. It was
designed to have a modular architecture so it can be used by a large number of industries from manufacturing to finance. 

The project was designed to be modular so all the components can be plugged in/out. The list of components (according to \cite{fabric-wiki}) is formed of:
\begin{itemize}
	\item an \emph{ordering service} that makes sure that the order of transaction is correct and that all peers receive the blocks of transactions via broadcast; the ordering service can be plugged in/out and it runs in an independent environment (e.g Docker container named \emph{hyperledger/fabric-orderer});
	\item a \emph{membership service provider} that enables only authorized entities to take part in the network(e.g a Docker container named \emph{hyperledger/fabric-ca}); any entity that wants to connect to the business network has to have a cryptographic identity created by the business network Certification Authority (CA);
	\item \emph{smart contracts} are applications that run independently in docker containers or other isolated environments and store agreements between network members; these smart contracts are named \emph{chaincode} and they can be written in almost any programming language;
	\item a variety of \emph{DBMS} options to store the ledger; a common option is CouchDB that runs in a Docker container named \emph{hyperledger/fabric-couchdb};
	\item more application's constrains for approval and validation policy;
	\item an optional \emph{peer-to-peer gossip service} to leverage messages to all the peers by broadcast; makes sure that everyone receives up-to-date information about the ledger transactions and fills in any gaps if one node missed some of the broadcasts.	
\end{itemize}

The chaincode (smart contract) is an application that facilitates and verifies the execution of a digital agreement between one or more network participants. The ledger is responsible for executing the digital contract at the time that it need to be applied. In a business network one or more digital contracts can co-exist and everyone that is connected to that network can create a contract. The network is responsible for validating that application code and it should be consider invalid until it is validated using order-execute architecture. In most Blockchain systems the smart contracts follow the rule order-execute so to reach consensus the order-execute design should be deterministic. Building a deterministic architecture  is quite challenging and the code should be written in domain-specific programming languages (e.g Ethereum developed a DSL - Solidity for their decentralized applications) in most of the cases.
Additionally this architecture brings performance issues due to the fact that transactions are added sequentially in the ledger.

The Hyperledger Fabric tries to solve the overhead of using the order-execute approach by proposing a new architecture type - execute-order-validate. This new design is composed by three stages. In the first stage the transactions are checked for correctness and if so they are immediately executed. After that, they are ordered using a consensus protocol and in the last stage they are validated according to an endorsement policy and added to the ledger. As a result of this approach the Hyperledger Fabric solves the issue of using DSL and non-determinism by removing any inconsistency and reduces the performace overhead. By removing any non-deterministic problem, the smart contracts can be written in any standard programming language. The first supported languages are Go and Node.js but the board of Hyperledger plans to support more popular programming languages such as Java in the future releases.

Permissioned Blockchain refers to restricting the access to the network to a known group of participants that are worth it to be trusted. It is not necessary that participants may know each other but all of them should be vetted by a trustful authority. The Fabric is using a permissioned Blockchain system so each member of the business network can easily be identified and if his/her actions (e.g deploying a smart contract or changing the configuration of the network) are malicious that member can be made accountable for them. In a permissionless Blockchain system it is very hard to identify the real person that stood behind a certain action so Hyperledger Fabric is more suitable for the business world. The Certification Authority is one instrument that can vet for the business network members and any respectful business can easily obtain the authorization to participate in the network.

Privacy and confidentiality is key for most of the business use-cases therefore any solution used has to guarantee that the data cannot be available to the public. 
In permissionless systems the transactions' data is available to every node in the Blockchain network so such a system fails to fulfill the privacy business requirements. The main approach of permissionless Blockchain systems is to encrypt their data. However, this solution is far from perfect because other organizations can hold enough computational power to break any encryption in a reasonable amount of time.
The approach of permissioned systems is to restrict the distribution of private data to only the involved participants via a channel. The Hyperledger Fabric allows any member of the network to create channels for different purposes so they can share confidential data with the authorized members.
Another solution that can solve this issue for both for permissionless and permissioned Blockchains is the Zero Knowledge Proofs (ZKP) protocol but it is still under research so in the meantime creating a channel to share data with authorized entities is a good alternative.  

As mentioned above, the Hyperledger Fabric offers support for pluggable components for many of its features. The ordering service is a pluggable module that is responsible of achieving consensus of ordering transactions. In the literature, there are two popular consensus protocols: Crash Fault Tolerant (CFT) and Byzantine Fault Tolerant (BFT). Each of them can have one or more implementations.
For example, the Hyperledger Fabric has two possible implementations for the CFT algorithm: Kafka (based on Zookeeper) and \emph{etcd} library from Raft.

The performance of Hyperledger Fabric depends on a variety of parameters: the number of nodes of the network, the number and size of transactions and hardware limitations so it is difficult to generalize the performance of the Hyperledger Fabric project. 
In order to determine the performance of a Business Network based on Fabric, the Hyperledger team has developed another project called Hyperledger Caliper. The community is contributing to define the set of measures that affect performance and scalability to help Caliper team build a powerful framework for benchmarking. 

Taking all these into account, the Hyperledger Fabric is one of the best solution for permissioned distributed systems that integrates the Blockchain technology.
The Hyperledger community is really involved in improving and building strong solutions for Blockchain business networks and the Hyperledger Fabric project has the biggest support from the community. Fabric supports a wide range of industries: from banking, manufacturing, supply chain to retail, healthcare, etc.
%\fi

\section{Hyperledger Composer}
\label{sub-sec:chapter1-section3}
The Hyperledger Composer framework is one of the assistive tools developed by Hyperledger team that focuses on helping developers accelerate the process of creating Blockchain systems for business.
Building Blockchain applications from scratch require high technical skills from networking protocols, cryptography and consensus algorithms and the development period can extend to one or more years. Having these issues in mind, Hyperledger community started a project to reduce the time and the skills set required to build Blockchain based systems. The project is called Hyperledger Composer and it is really appreciated by developers by reducing the time and effort from years to months.

The Hyperledger Composer framework can be easily integrated with Fabric via a Loopback REST API. Before getting into more details on how we can integrate the two projects it is important to describe the idea behind Hyperledger Composer.

This framework is using a Business Network Archive (BNA) to store information about the participants of the Blockchain network, the assets they can store in the ledger via transactions. Each business requires a custom BNA which defines the type of members, their roles and permissions and the transactions that can be done in the Blockchain. For example, in the supply-chain industry we need one or many manufacturers, buyers/clients, distributors and regulators. All these participants have restricted access to submit transactions or change the network configurations based on their role. For example a buyer can submit an \emph{order} transaction but he/she cannot submit a \emph{delivery} transaction.

The BNA has a well defined structure and each component should be defined using a custom Composer syntax.
According to Hyperledger Composer Community \cite{composer-wiki} the BNA has to follow this structure:
\begin{itemize}
	\item one or more Model files - written using Hyperledger Composer Modeling Language, an object oriented language that enables developers to define three elements: the namespace (only one for all the resources declared in the model), the list of entities of the network: participants, assets, events and transactions and the possibility to add entities defined in other namespaces;
	\item one or more Transactions Functionalities file - each transaction should have a metadata, decorators and a JavaScript function; these functions run when a participant submits a transaction; the REST API exposed by Hyperledger Fabric is invoked in the transaction function via \emph{getNativeAPI} and the Hyperledger Composer API via \emph{getAssetRegistry}; also these functions can issue events that other external systems can listen to via the \emph{composer-client} library;
	\item an Access Control role file -  written using Hyperledger Composer Access Control language that enables setting permissions for all the entities defined in the model; the rules that can be enabled/disabled are Create Read Update and Delete (C.R.U.D.); there are two types of access controls: business and network; both types are defined in the Access Control files and the business control  is applied after the network control;
	\item a Query Definition file - it is an optional component that requires the \emph{composer-rest-server} to be up and running; there are two types of queries: named and dynamic; the named queries are static, defined in the BNM and exposed by the REST server via GET endpoints; the dynamic queries can be defined in transaction functions or directly in the client application.
\end{itemize}  
The \labelindexref{Figure}{img:composerBna} outlines the BNA structure with all the components and a brief description for each of them.
\fig[scale=0.6]{src/img/composerBna.PNG}{img:composerBna}{Hyperledger Composer Business Network Archive}.

There two possible ways to use this framework: the playground and the CLI tools. The first solution offers a web user-interface to define, deploy and test business network directly on cloud or locally on the personal machine. The other solution is for Blockchain developers that want access to all the Fabric functionalities and that can be easily integrated with Visual Studio Code IDE and Atom editor.

More than that, the Hyperledger Composer has a specialized storage mechanism which decouples transactions information from any other information such as participants and assets data. All transactions are stored in the Blockchain ledger, everything else is located in the state database associated with that ledger. Another interesting feature is the Historian Registry that stores the transactions as \emph{HistorianRecord} assets of the business network. These assets' definition is available in the Hyperledger Composer Model and they store information about the transaction that created them, the identity of the participant and the participant that invoked the transaction.

Another concept related with the Composer framework is the Connection Profile. Each Blockchain Network has a Connection Profile that defines the system and it is created by the network administrator to enable members to connect to the system. The Hyperledger Composer uses the Identity concept to ensure that only authorized members connect to the network. An Identity has a private key, a digital certification and a metadata file that is optional and may store the network name and other details about it. A Connection Profile and an Identity form a Business Network Card that offers a member access to the network. It is worth mentioning that only one identity can be stored in a Business Network Card and that identity may be associated with one of the participant that exists in the network. The concept of Business Network Cards was introduced to enable groups of participants with the same access rights to access the network so it is related with the network access control management.



 
