\chapter{Hardware Supply-Chain Network}
\label{chapter:chapter4}

Many software applications developed by Ixia Solutions Group (ISG) teams run on dedicated hardware and in many situations only one employee can use that hardware at a time.
The existing Capital Expense (CAPEX) procedure is very complicated and the time between ordering a new device and receiving it can be really long. It may happen that the team doesn't need it anymore at the time the device was received. The dedicated hardware is expensive (some chassis may cost up to \$50000) so not receiving it on time may cost the department a lot of money and it can also delay the releases.

On each quarter, the manager has the responsibility of determining what are the hardware needs of the team and send the list to her/his superior for approval.
Also, the manager has to check the price of the devices requested and if the costs are over the CAPEX budget, the hardware list should be prioritized. 
After completing the list and set the priorities if needed, the manager has to fill an \emph{Internal Sales Order} on an internal platform and than wait for the device to be delivered.

A huge disadvantage of this procedure is the lack of transparency regarding the status of the order and the time estimation until receiving the order.
In addition to this the costs may differ due to the international tax changes. 

The \textbf{\emph{{\project}}} project is a supply-chain business network that solves both the transparency and the access control restrictions.
This system is using Hyperledger Composer framework for modeling the business network: participants, transaction and assets and connects to the Hyperledger Fabric via its API to add transactions in the Blockchain database and to manage participants' access accordingly with their roles in the network.
Therefore the Hyperledger Fabric is suitable for solving the hardware devices supply chain transparency and bottleneck issues and to restrict the access of the participants involved in accordance with their position in the company.


\section{ISG CAPEX Procedures}
\label{sub-sec:chapter4-section1}
The \emph{Hardware Supply-Chain} use cases are: ordering a new hardware device and sending the device for service investigation(RMA).
Ixia Solutions Group (ISG) teams are focused on developing software applications for testing the network performance, getting network visibility by monitoring the traffic or finding security threats before compromising any data.
Therefore teams need specialized hardware devices such as routers, gateways, switches to more powerful machines as packet brokers, ESXI servers, clusters, etc.
The ISG division offers deep expertise in network testing, security and visibility which requires investing in powerful hardware equipment.
In each quarter of the financial year the ISG division has a CAPEX budget to invest in hardware so each team can request new hardware devices.
Each manager has the responsibility of identifying the hardware requirements by consulting his/her team and assuring that the total cost doesn't exceed the budget in which case the CAPEX list should be prioritized.

When the CAPEX requirements list is completed, the manager can proceed to order them.
For ordering the equipment he/she has to send emails for approval and after the order is accepted by a superior, the manager can added it in an internal platform.

The RMA procedure can be performed by the team manager as soon as he/she is informed about the hardware malfunction. Most hardware devices have an warranty period where they can be replaced or repaired free of charges by the manufacturing divisions. Before sending it to RMA the employee has to get approval from his/her manager who also needs to ask for approval from the superior.

Taking all these into consideration, the CAPEX can become quite exhausting and if something happens with the hardware it is hard to find the root cause of the issue. In order to solve the issue of transparency when tracking a device and to reduce the time of getting all the required approvals we propose a Blockchain based solution that allows everyone to check the status of orders and hardware devices. More than that we want a solution that offers immutability so nobody can modify the history of a chassis which is one of the core features of the Blockchain protocol.
The idea is to build a permissioned Blockchain application that solves the transparency and the immutability requirements and is able to offer support for the network access control problem.
For reasons of competitiveness a business has to keep its internal procedures private and it is important to use a solution that enables only authorized members to get access to the system data.
After some research work we found two frameworks developed by Hyperledger community specially for these business requirements so we decided to use them to build our hardware supply-chain system that can be used to replace the CAPEX complicated procedures.

\section{Hardware Supply-Chain Use Cases}
\label{sub-sec:chapter4-section1}
The hardware supply-chain offers support for two main uses cases. The first use case solves the new hardware ordering procedure and the second use case solves the situation when a device is due to service (RMA).

In order to prove the functionalities of this system we have created instances for all types of participants and three types of network card for each of them. 
In \labelindexref{Figure}{img:playground} we can see the three network cards used by the three types of participants: \emph{employee@keysight}, \emph{hardware@keysight} and \emph{regulator@keysight.com}.

\fig[scale=0.3]{src/img/playground.png}{img:playground}{Hardware Supply-Chain: Playground} 

The first use case flow is triggered by an employee by creating an asset of type \emph{Order} via a transaction of type \emph{PlaceOrder} (see \labelindexref{Figure}{img:placeOrder1}). 

\fig[scale=0.4]{src/img/placeOrder1.png}{img:placeOrder1}{Hardware Supply-Chain: Example of \emph{PlaceOrder} Transaction} 

Each order has assigned a manufacturer that is in charge and authorized to create the device.
The manufacturer analyzes the order and checks if the requested device is available or need to be scheduled for manufacturing. If no device is available the manufacturer submits a transaction of type \emph{UpdateOrderStatus} and update the order status in \emph{SCHEDULED_FOR_MANUFACTURE} (see \labelindexref{Figure}{img:scheduled1}). 

\fig[scale=0.4]{src/img/scheduled1.png}{img:scheduled1}{Hardware Supply-Chain: Update order status in \emph{SCHEDULED_FOR_MANUFACTURE}}

When he finished creating the hardware ordered, he/she sets a serial number for the device by submitting a transaction of type \emph{UpdateOrderStatus} with the status \emph{SERIAL_NUMBER_ASSIGNED} and a unique value for the field \emph{serialNumber} (see \labelindexref{Figure}{img:serialNumber1}).

\fig[scale=0.4]{src/img/serialNumber1.png}{img:serialNumber1}{Hardware Supply-Chain: Update order status in \emph{SERIAL_NUMBER_ASSIGNED}}

In this moment, a new chassis is added in the list of assets stored in the Blockchain that can be easily identified via its serial number (see \labelindexref{Figure}{img:chassis1}).

\fig[scale=0.5]{src/img/chassis1.png}{img:chassis1}{Hardware Supply-Chain: The manufactured chassis}

In the end, the manufacturer has to send the device to the employee that ordered it. This step is completed by submitting a transaction of type \emph{UpdateOrderStatus} and change the order status to \emph{DELIVERED} (see \labelindexref{Figure}{img:delivered1}). The manufacturer is also responsible to correctly inform the delivery company all the legally required details about the packet and to set the recipient of the packet to be the direct manager of the employee that ordered the device. 
This use case flow ends when the designated manager receives the hardware and assigns it to the employee that ordered it via a transaction of type \emph{UpdateOrderStatus} (see \labelindexref{Figure}{img:ownerAssigned1}). So the order changes its status in \emph{OWNER_ASSIGNED} and the chassis has a new property set \emph{owner} (see \labelindexref{Figure}{img:chassisOwner1}).
 
\fig[scale=0.4]{src/img/delivered1.png}{img:delivered1}{Hardware Supply-Chain: Update order status in \emph{DELIVERED}}

\fig[scale=0.4]{src/img/ownerAssigned1.png}{img:ownerAssigned1}{Hardware Supply-Chain: Update order status in \emph{OWNER_ASSIGNED}}

\fig[scale=0.5]{src/img/chassisOwner1.png}{img:chassisOwner1}{Hardware Supply-Chain: Chassis' Owner}

All the submitted transactions are registered and stored in the Blockchain ledger so nobody can corrupt that data. Having an immutable ledger is one of the powerful advantages of the Blockchain usage. The Hyperledger Composer playground offers also the possibility to read the information stored in the ledger (see \labelindexref{Figure}{img:history1}).
\\
\fig[scale=0.3]{src/img/history1.png}{img:history1}{Hardware Supply-Chain: Blockchain Ledger}

The second use case is similar with the one described above with the mention that the stage of delivery may be missing. In this situation the status of a chassis is set to \emph{SCRAPPED}.

The user interface was provided by Hyperledger Composer playground but the business model we defined can be integrated with other web frameworks via the \emph{composer-rest-server} module. The business network can be deployed on this server and we can connect from any web client to the REST API it has exposed.
