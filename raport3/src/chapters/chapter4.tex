\chapter{Project Use Case}
\label{chapter:chapter4}

\section{Use Case Description}
\label{sub-sec:chapter4-section1}
We choose to demo a business use case for the field "Evidence Tracking and Documents" starting with downloading the business network model from \project application and deploying the \emph{.bna} file to the Hyperledger Fabric.

For this demo, we created a virtual machine that runs a Linux OS (Ubuntu 18.04) with a full environment configured. For simplification, both the \project and the Hyperledger Composer run on the same VM. 

We want to create a system that helps business organizations and universities store the students' profile. Having a Blockchain solution that stores data about students such as their courses and studies level.
The idea is to offer the participants of the network information about students based on their role in the business network.
For example, the university can access information about students' degrees and personal data, the student has full access to his/her profile and external institutions such as other universities or companies can access only information that the student agrees to display. One possible information that a student may want to share with companies can be the level of study and the courses he attended to or other certifications.

Such a system guarantees that nobody can change the student's history and simplifies the process of validating diplomas. 
Therefore, it is useful for all the network participants:
\begin{itemize}
	\item for students that want to transfer to another university or that want to prove that they have the skills to get the desired job
	\item for universities by improving the time and costs of students' data collecting and transferring to other academic institutions
	\item for companies and organizations that want to validate the resume information they receive
\end{itemize}

\section{Use Case Steps}
\label{sub-sec:chapter4-section2}
The use case described above requires executing the following steps:
\begin{itemize}
	\item download a business network model suitable for the business field chosen
	\item deploy the business network archive to the Hyperledger Composer
	\item check that all the business logic and  business network definition is correct for the business use case
	\item add some test data about student/s and do some testing to check that the business network is correctly defined; if you are play the role of the student you should have full access to the data from the Blockchain ledger; if you are the student' university or other institution the access level should be the one defined by the student in the \emph{Access Control} file from the \emph{.bna}	
\end{itemize}

From the web application we can download a sample of the \emph{.bna} file for Identification Profiling and then update it for the current scenario. (see \labelindexref{Figure}{img:bnm})
\fig[scale=0.2]{src/img/bnm.png}{img:bnm}{Business Network Modeling Dashboard} 

For the selected category \emph{Documents and Evidence Tracking} we can download one of the two samples available: \emph{Identification Profiling} or \emph{Digital Property}(see \labelindexref{Figure}{img:evidence}).

The downloaded archive contains the files described in section \ref{sub-sec:chapter1-section2}.

\fig[scale=0.2]{src/img/evidence.png}{img:evidence}{Documents and Evidence Tracking Page} 

In the \emph{.bna} file we can see the business network definition that has the following structure:
\begin{verbatim}
**Participant**
`Student`
`University`
`Organization`

**Asset**
`StudentProfile`

**Transaction**
`StudentProfileTransaction`

**Event**
`StudentProfileEvent`
\end{verbatim}

\emph{StudentProfile/s} are owned by \emph{Student/s}, and the value property on a \emph{StudentProfile} can be modified by submitting a \emph{StudentProfileTransaction}. The \emph{StudentProfileTransaction} emits a \emph{StudentProfileEvent} that notifies applications of the old and new values for each modified \emph{StudentProfile}.	
The list of \emph{Student} participants is defined in \labelindexref{Figure}{img:student}.
The list of \emph{University} participants is defined in \labelindexref{Figure}{img:university}.
The list of \emph{Organization} participants is defined in \labelindexref{Figure}{img:organization}.


\fig[scale=0.2]{src/img/student.png}{img:student}{Participant - Student} 
\fig[scale=0.2]{src/img/university.png}{img:university}{Participant - University} 
\fig[scale=0.2]{src/img/organization.png}{img:organization}{Participant - Organization}

After defining the participants we have to create the assets. In our example, the asset is the \emph{StudentProfile}(see \labelindexref{Figure}{img:studentprofile1}).
\fig[scale=0.2]{src/img/profile1.png}{img:studentprofile1}{Asset - StudentProfile}

To test the Hyperledger Composer framework, we submitted a transaction (see \labelindexref{Figure}{img:transaction}), where we changed the value of a \emph{StudentProfile}. The new value can be seen in \labelindexref{Figure}{img:studentprofile2}.

\fig[scale=0.3]{src/img/transaction.png}{img:transaction}{Submit a StudentProfileTransactions}.

\fig[scale=0.3]{src/img/profile2.png}{img:studentprofile2}{Asset - Updated data of a StudentProfile}


