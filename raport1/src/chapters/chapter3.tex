\chapter{Blockchain based projects}
\label{chapter:chapter3}
Don and Alex Tapscott \cite{blocknextinternet} define Blockchain as \emph{an incorruptible digital ledger of economic transactions that can be programmed to record not just financial transactions but virtually everything of value}.

The Blockchain technology has a huge potential of development and the number of areas that it could innovated is impressive. In \cite{blocktechguide}, Ameer Rosic gathers a complete list of fields that could be innovated by Blockchain technology.
\section{Financial}
\label{sub-sec:chapter2-subsection1}
Blockchain and the financial world involve trading, money transfer and mortgages. Right now the economic world releases on banks, governments and other
financial institutions to transfer money from an account to another. Financial information transfer is monitored by governmental institutions and this type of information can be used to banks advantages in trading operations.

\subsection{Cryptocurrencies}
\label{subsec:cryptoprojs}
The Bitcoin project is a popular blockchain based application but there are other blockchain projects that deserves to be brought to attention such as Ethereum, Namecoin or Ripple.
The Ripple project started before Bitcoin in 2004 and it passed through many changes over time. Today, Ripple is a solution that enables banks and clients to exchange value. Similar with Bitcoin, Ripple is a distributed system with its own blockchain and a native currency called ripple.
Ethereum is another blockchain based application that has its own Turing-complete programming language which makes it more powerful than Bitcoin by allowing users to create their applications on a new programming language.
Namecoin was the first fork of the Bitcoin project. It offers a domain name registry similar to the Internet’s DNS. It is an alternative domain name service for the root domain \emph{.bit}.
The project has its own currency called namecoin (symbol \emph{NMC}) which is used to pay transactions fee for registration, update and transfer of domains name.

\section{Smart contracts}
\label{sec:smartcontracts}
The smart contracts are applications that run as independent nodes in the Blockchain network and execute actions when special conditions are accomplished. Ethereum is an open-source project that have its own operation system and programming language that have the potential to leverage the usefulness of Blockchain projects. Using Ethereum we can program a smart contract to pay out a derivative when a financial instrument reaches a certain benchmark and the payout is automated.

\section{Sharing services}
\label{sec:sharingeconomy}
The sharing economy is a successful field in the modern world. Companies such as Uber and AirBnB are flourishing and users are becoming interested in sharing services. The change that can be made by the use of Blockchain is to remove the power from centralized systems such as Uber or other platforms that allows sharing by using a peer-to-peer network for sharing services. An example of such a system is OpenBazaar that is a Blockchain project for a eBay peer-to-peer.


\section{Crowdfunding and Voting Systems}
\label{sec:crowdfunding}
Companies like Kickstarter and Gofundme are examples of Blockchain projects that are creating crowd-sourced venture capital funds for different projects. The popularity  of these companies highlights that people are willing to get involved in product development and make a difference in a challenging industry such as IT\&C.
Another experiment is the Ethereum-based DAO (Decentralized Autonomous Organization) that raised \$ 200 million USD in a few months in \emph{DAO} tokens. People bought tokens to vote on smart contract venture capital investments. The project was hacked due to insufficient diligence with disastrous consequences.

\section{Public sector}
\label{sub-sec:chapter2-subsection2}
The public sector activities can benefit from blockchain technology by changing the asset registration, citizen identification process, medical records and
medicine supply chain. The real power of blockchain consists of creating new
solutions of performing old processes that are inefficient in the modern world.
Checking paper records for validating the ownership of an asset had become very
difficult and time consuming activity. The number of population is increasing
at a high rate and public sector employees are forced to perform lucrative and
repetitive tasks. Having all properties stored in a blockchain repository eliminates the problems mention above and offers the guarantee of security and
privacy.

Another problem of the public sector is protecting critical data. Despite governmental institutions effort to protect their systems, criminals manage to get access to their databases and manipulate their records.
Using encryption methods to protect personal information is 100 percent safer than storing data in DBMS.
There is an application called Keyless Signature Infrastructure (KSI) that uses Blockchain to protect all public sector data in Estonia. The KSI project uses hash values to identify records. The values are unique and there is no possible way to reconstruct the information for that record(file).
Right now, the electronic health records of all Estonia citizens uses KSI technology.

\section{Digital property ownership}
The process of owning and selling assets involves multiple interaction and a long paper trail. To improve this difficult process in Sweden, Lantmäteriet the government is exploring ways to digitize it. A mobile prototype app was launched that offers transaction mechanism for sellers and buyers as well as for real-estate agents and banks. The basic technology used is Blockchain that stores all information about existing properties and it will improve the communication among all the parties involved in the process. The time reduction is expected to be from three-to-six months to days or hours.
\section{Retail}
\label{sub-sec:chapter2-subsection3}
In the retail field blockchain can change the way the supply process, fidelity
programs are implemented or the information sharing process is done (from
supplier to retailer and the other way around).

\section{Insurance}
\label{sub-sec:chapter2-subsection4}
There are many activities in the insurance industry than can be innovated such as risk provenance, claims processing, asset usage history or claims files.

\section{Manufacturing}
\label{sub-sec:chapter2-subsection5}
Another industry that can benefit from blockchain technology is the manufacturing starting with the supply chain, product parts or maintenance tracking.
In tech and telecommunication manufacturing all big players vies for the IoT dominance. IBM, AT\&T and Samsung are researching to detect the best solution to predictive maintenance of mechanical part or data analytics. Monitoring these systems is hard and expensive but Blockchain is offering a cheaper solution for keeping track of IoT applications components.

\section{Prediction markets}
\label{sec:prediction}
Making prediction based on event probability was proven to have a high degree of accuracy. Augur is an example of sharing offering project for the prediction market that is based on the outcome of real-world events. The people that uses Augur can earn money buy buying into the correct predictions.

