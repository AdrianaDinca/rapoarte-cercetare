\chapter{Background}
\label{chapter:chapter2}

\section{Decision Support Systems Architecture}
\label{sec:ch3-sec}
Developing Decision Support Systems requires high costs, human effort, time and the success of the system cannot be guaranteed. These system can be affected by many risks like system design, data quality and old technologies. Their purpose is to assist managers in decision making such as investments, budgeting, cash flows or financial planning.

\textbf{Level 1 (bottom-tier) – data management}

Level 1 (bottom-tier) is the level of data management and it involves data, metadata, DBMS, data warehouses and marts, data dictionaries and metadata dictionaries.
The data that comes from operational databases and external sources is transformed using different types of filtering tools and predefined procedures.
At this level of the architecture a series of tasks should be implemented in order to load data in a data warehouse such as collecting and extracting from external data, data cleaning and transformation and than loading the data in the data warehouse.
The processes performed at this level are very important for the success of the DSS. A faulty design could lead to the failure of the DSS.

\textbf{Level 2 (middle-tier) – model management or analysis level}

The second level of the DSS architecture is involved in modeling and analysis. At this level the data is transformed and the necessary information is extracted. In order to get that information architects are using data analysis, simulation and forecast models that achieves the business requirements. An example of technology used at this level is OLAP. The OLAP tool is using multidimensional data representation that allows quick data analysis by using rool-up or slice operations.
Data mining techniques are present at this level so they are extracting knowledge from that data not reports of the data.
In \cite{39029} the authors state that 'very often, the success of a DSS is determined by the discovery of new facts and data correlations and not by building reports that presents data'.

\textbf{Level 3 (top-tier) – The interface or the presentation layer}

The level 3 contains the interaction between users and the system. The user interface should be specially designed to make the interaction more user friendly. This level involves queries and reports building tools, data viewing tools, data publishing and presenting tools. On this level, the human resource has an important role because he/she need to make the final decision. To facilitate the interaction between the system and the end-users the architects developed also smart portals using portal-based web technologies. BI portals offers users a good end-to-end experience by including nice graphics, report generation features, etc.

\textbf{Level 4 – Telecommunications}

The telecommunicationslevel interconnects all three levels and it contains web servers, computer networks, communication devices, etc.


\section{Blockchain technical details}
\label{sec:chapter2-section1}
The blockchain is a public ledger of all transaction sent through the network that is known by every one that holds a blockchain node. More precisely, each node has a copy of the blockchain in order to validate or invalidate a transaction. The majority of nodes decide if a transaction is valid and if so that transaction is added in the latest created block by one of the miners. The choice of using blockchain for data storing is related with the fact that the blockchain’s deep history is immutable so it ensures the protocol security.

The blockchain has no central authority for managing assets flow along the network. This technology is recognized as the “fifth evolution” of computing, the missing trust interface of the Internet as stated by Laurence~\cite{blockchaind}.

The blockchain structure has the following components as described in \labelindexref{Figure}{img:mt-diag}:
\paragraph{$\bullet$ The block}is an array of transactions recorded into a ledger. Depending on the blockchain the transaction has assigned a different type of value.
Each block has its own hash and a merkle tree with hashed transactions. Storing transactions in a merkle tree is a design decision that saves disk space. Nakamoto~\cite{bitcoinwhitepaper}\ explains the reason why using a merkle tree makes disk space recover possible. The explanation for this is that "once the latest transaction in a coin is buried under enough blocks, the spent transactions before it can be discarded to save disk space. To facilitate this without breaking the block’s hash, transactions are hashed in a Merkle Tree, with only the root included in the block’s hash. Old blocks can then be compacted by stubbing off branches of the tree. The interior hashes do not need to be stored" according to Satoshi~\cite{bitcoinwhitepaper}.
\paragraph{$\bullet$ The chain}is simply a hash that connects two blocks in chronological order. The hash for a new block is generated based on the data that was in the previous block. The hash algorithm used by Bitcoin is Secure Hash Algorithm 256~\cite{sha2} that creates an unique (collision probability is negligible), fixed-size (256 bits) hash. 
\paragraph{$\bullet$ The network}is represented by full nodes. We can think of it as a computer that runs an algorithm for securing the network.
Mining is expensive and it requires a lot of computer power so the blockchain algorithm rewards miners for their service. The payment is usually a currency or a token.

\fig[scale=0.5]{src/img/mt-diag.pdf}{img:mt-diag}{Blockchain Structure using Merkle Tree}


\section{Blockchain properties}
\label{sec:chapter2-section2}
\subsection{Proof-of-work}
\label{subsec:proof-of-work}
The idea of using a proof-of-work~\cite{proofofwork} system to legitimate the user of an application was used before in other software systems.
For example, Hashcash~\cite{hashcash} is a proof-of-work system introduced by Black that was invented to limit email spam and denial-of-service attacks. The idea is quite easy to follow: a legitimate sender needs to spend a reasonable amount of computing time in order to issue an email. If a legitimate user sends a reasonable number of emails, a spammer wants to send thousands of emails making the spamming effort very expensive. The receiver’s role is to validate the hash, which is quite easy. Hashcash is conceptually similar to the proof-of-work system used by Bitcoin miners.

Solving the problem of double spending using a proof-of-work system provides security guarantees for naming/identity registration and transfer.

\subsection{Consensus Protocol}
\label{subsec:overview-lbl}
Based on prior work, the Bitcoin technology introduces a consensus protocol based on the Byzantine Generals’ Problem~\cite{byzantine} to eliminate double-spending by transaction validating process.

When a new transaction is broadcasted to the network, nodes have the option to add the transaction in their copy of the Blockchain or not. The majority of nodes which compose that network decides the single state of a transaction (valid or not) and the consensus is achieved.


\section{Bitcoin project}
The Bitcoin is the first cryptocurrency developed using blockchain technology. The unique attributes of the blockchain and the cryptographic principles
brought Bitcoin a huge popularity. Satoshi Nakamoto is the pseudonim of the
Bitcoin inventor. The real identity of Satoshi was never revealed. There are
many suppositions but not of them was confirmed. He claimed to be a Japanese
male around forty years old.

One interesting aspect about Bitcoin technology is how it solves both the problem of double-spending and mining in a decentralized system. The protocol is developed on “sound cryptographic” principles that guarantee proof-of-work, proof-of-ownership and classic currencies attributes such as fungibility and scarcity.

